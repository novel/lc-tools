\documentclass[a4paper]{report}
\usepackage{verbatim}
\usepackage{hyperref}
\title{An lc-tools Tutirial}
\author{Roman Bogorodskiy\\
\texttt{novel@FreeBSD.org}}
\date{\today}
\begin{document}
   \maketitle
   \tableofcontents
   \section{Introduction}
   An {\tt lc-tools} is a set of command line tools to manage various Cloud (aka IaaS)
   Providers. It's written in \href{http://www.python.org}{Python} and 
   uses \href{http://libcloud.org}{libcloud} to interact with provider's API.xx
   
   \section{Getting Started}
      \subsection{Installing}
          \subsubsection{Dependencies}
          The only external dependency is {\tt libcloud}. Please visit 
          \href{http://incubator.apache.org/libcloud/downloads.html}{libcloud download page}
          to get information how to download and install it.

          \subsubsection{Installing from PyPI}
          Latest stable version of {\tt lc-tools} could be installed from PyPI:

          \begin{verbatim}
           easy_install lctools
          \end{verbatim}

          \subsubsection{Installing from source}
          To get latest development version you can checkout sources from 
          \href{https://github.com/novel/lc-tools}{project's github page}:

          \begin{verbatim}
           git clone https://github.com/novel/lc-tools.git
          \end{verbatim}

          \textbf{Note!}: latest development version requires latest {\tt libcloud}
          sources!
      \subsection{Configuring} \label{conf}
      All configuration data is stored in text file {\tt ~/.lcrc}. It doesn't exist
      by default so you have to create it:

      \begin{verbatim}
      touch ~/.lcrc
      chmod 600 ~/.lcrc
      \end{verbatim}

      We need {\tt chmod} to be sure others can't access you secret keys. If file mode
      will be too permissive, {\tt lc-tools} will not run.

      Structure of the file is following:

      \begin{verbatim}
      [default]
      driver = foogrid
      access_id = your_key_id
      secret_key = your_password

      [barcloud]
      driver = barcloud
      access_id = somekey
      secret_key = some_key
      \end{verbatim}

      You probably noted that config file is separated by sections (\textbf{default},
      \textbf{barcloud} in the example above). Each section corresponds to a signle
      cloud account, it's possible to switch them and \textbf{default} section is
      used by default and should always be present.

      All fields are necessary and cannot be omitted.

      \begin{description}
        \item[driver] \hfill \\
          name of the specific cloud provider for the account. To get a listof all
          available drivers please execute

          \begin{verbatim}
            lc-drivers-list|cut -d "." -f3|sort|uniq
          \end{verbatim}

        \item[access\_id] \hfill \\
          access\_id for the account. It could be either login or some generated key
        \item[secret\_key] \hfill \\
          secretkey for the accout, in other words it's a password for given access\_id
      \end{description}

      Please refer to your cloud prover for detailed documentation on getting account's
      credentials.

      \section{Basic Usage}
         \subsection{Glossary}
            Various cloud providers use different terms like images, flavours, nodes,
            servers etc. We will just use terms used in libcloud internally. 

            \begin{description}
               \item[Node] \hfill \\
               Single instance of \textit{virtual server} on the cloud. 
               \item[Image] \hfill \\
               Understand it as a \textit{server template}, the base from your nodes
               will deliver from.
               \item[Size] \hfill \\
               Size of the \textit{node}, i.e. hardware features of \textit{node}. 
               It could imply RAM, disk space and so on.
            \end{description}
         \subsection{General Info}
         Lc-tools package consists of several tools each of them does its own thing:
         listing images, creating nodes, etc. An easy way to learn what tools exist
         type \texttt{lc-} in your shell's prompt and press Tab key to autocomplete.

         Every tools allows to switch configuration profile (please refer 
         to \nameref{conf} if you don't know what it is). This is done using 
         \texttt{-p} switch like this:

            \begin{verbatim}
            lc-node-list -p myacc2
            \end{verbatim}

         Where \texttt{myacc2} is a name of profile you want to use. Again, this
         feature works for every tool, not only \texttt{lc-node-list}.

         \subsection{Listing Images}
            You can list available images using \texttt{lc-image-list} tool. Here
            is a sample of its output:

            \begin{verbatim}
         $ lc-image-list|grep -i centos
         image CentOS 5.2 (32-bit) w/ RightScale (id = 62)
         image CentOS 5.2 (64-bit) w/ RightScale (id = 63)
         image CentOS 5.3 (32-bit) w/ None (id = 1531)
         image CentOS 5.3 (64-bit) w/ None (id = 1532)
         $
            \end{verbatim}

            The first line is a shell command we've issued. I've piped it to
         \texttt{grep} to output only images containing 'centos' it their names,
         otherwise the list would be too long. 

         For every image you see its name and id, for example for 
        \textit{CentOS 5.3 (64-bit) w/ None} id will be \texttt{1532}. Please
        remember it because you will need it if you're going to use this
        image.

        \subsection{Listing Sizes}
        Sizes can be viewed using \texttt{lc-sizes-list} tool. An example of its
        output for GoGrid:

        \begin{verbatim}
        $ lc-sizes-list
     size 512MB (id=512MB, ram=512, disk=30 bandwidth=None)
     size 4GB (id=4GB, ram=4096, disk=240 bandwidth=None)
     size 2GB (id=2GB, ram=2048, disk=120 bandwidth=None)
     size 8GB (id=8GB, ram=8192, disk=480 bandwidth=None)
     size 1GB (id=1GB, ram=1024, disk=60 bandwidth=None)
       $
       \end{verbatim}

       Please note that ids are not always numeric as could be seen by this example.
       So, if we want to create, say, 2GB server, we should use id \texttt{2GB}.

       \subsection{Node Creation}
       Now as we've learned how to observe available images and sized we can proceed
       with node creation using \texttt{lc-node-add} tool.

       It's as simple as:

       \begin{verbatim}
       $ lc-node-add -i 62 -s 1GB -n mynewnode
       $
       \end{verbatim}

       Here we created a new node with image \texttt{id = 62}, size
       \texttt{id = 1GB} and name \texttt{mynewnode}.

       In the next section we will discuss how to list nodes to make
       sure your new node created and obtain details about it.

       \subsection{Listing Nodes}
       Getting list of nodes is not harder than getting list of images
       or sizes:

       \begin{verbatim}
       $ lc-node-list
       100xxx  mynode1     173.204.xx.yy  Running
       100xxx  mynode2    173.204.xx.zz  Running
       $
       \end{verbatim}

       Format of this output is following:

       \begin{center}
         \begin{tabular}{ | c | c | c | c | }
           \hline
             image\_id & name & public\_ips & state \\
           \hline
         \end{tabular}
       \end{center}

       Where:

       \begin{description}
          \item[image\_id] \hfill \\
          Id of the node automatically generated by cloud provider
          \item[name] \hfill \\
          Name of the node you gave at creation.
          \item[public\_ip] \hfill \\
          Comma-separated list of public ips assigned to the node.
          \item[status] \hfill \\
          Status of the node, helps to understand if node is usable or not.
       \end{description}

       \subsection{Operations on Individual Nodes}
       Tool called \texttt{lc-node-do} allows to operate on individual nodes.
       Currently, it allows rebooting and destroying (deleting) nodes.

       It can be done this way:

       \begin{verbatim}
        $ lc-node-do -i 123 destroy # for deleting
        $ lc-node-do -i 124 reboot  # for rebooting
        \end{verbatim}

       Here an arugment for \texttt{-i} switch is an id of the node we're working
       with and the next arugment is an action, i.e. what we want to do with the node.

       It's possible to specify more than one node id at time, for example:

       \begin{verbatim}
        $ lc-node-do -i 10,34,98 destroy
        $
        \end{verbatim}

       This command will destory nodes with ids: 10, 34 and 98. Also, it's possible to
       specify ranges of ids like that:

       \begin{verbatim}
        $ lc-node-do -i 100-119 destroy
        $
        \end{verbatim}

      This will destroy nodes with ids starting from 100 and ending with 119.

\end{document}
